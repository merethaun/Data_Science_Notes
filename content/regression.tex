\section{Regression}

In the next three chapters, we'll look at \textbf{error-based learning}\sidenote{Error-based learning}. In general, this learning approach functions as follows:
\begin{itemize}
  \item We have a \textbf{parameterized prediction model} which is initialized with random parameters.
  \item An \textbf{error function}\sidenote{Error function} is then used to evaluate the performance of this model when it makes predictions for instances in a training dataset.
  \item Based on the results of the error function, the parameters are \textbf{iteratively adjusted} to create a more and more accurate model.
\end{itemize}

There are different approaches to realizing error-based learning:
\begin{itemize}
  \item Regression (covered in this section)
  \item SVMs (covered in the next section)
  \item Neural networks (covered in a later section)
  \item Genetic algorithms, or other evolutionary approaches
\end{itemize}

We'll start with \textbf{regression}\sidenote{Regression} and the following basic idea. 
\begin{align*}
  \text{Our model: } & f : \text{\color{mathblue}descriptive features} \rightarrow \text{\color{mathblue}target features}\\
  \text{Goal: } & \text{find } f \text{ minimizing } error(\text{\color{mathblue}prediction}, \text{\color{mathblue}observed data})
\end{align*}
When we compare this approach to decision trees, we see:
\begin{itemize}
  \item Decision trees were initially developed for categorical features and then extended to continuous features.
  \item Regression followed the reverse path, which means it's most \textbf{suitable for continuous data}.
  \item Still, both are supervised learning techniques.
\end{itemize}

\subsection{Simple linear regression}


\subsection{Multiple descriptive features}
\subsection{Interpretation of results}
\subsection{Hanlding categorical features}
\subsection{Logistic regression}
\subsection{Extensions (non-linear and multinomial)}

