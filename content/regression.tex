\section{Regression}

In the next three chapters, we'll look at \textbf{error-based learning}\sidenote{Error-based learning}. In general, this learning approach functions as follows:
\begin{itemize}
  \item We have a \textbf{parameterized prediction model} which is initialized with random parameters.
  \item An \textbf{error function}\sidenote{Error function} is then used to evaluate the performance of this model when it makes predictions for instances in a training dataset.
  \item Based on the results of the error function, the parameters are \textbf{iteratively adjusted} to create a more and more accurate model.
\end{itemize}

There are different approaches to realizing error-based learning:
\begin{itemize}
  \item Regression (covered in this section)
  \item SVMs (covered in the next section)
  \item Neural networks (covered in a later section)
  \item Genetic algorithms, or other evolutionary approaches
\end{itemize}

We'll start with \textbf{regression}\sidenote{Regression} and the following basic idea. 
\begin{align*}
  \text{Our model: } & f : \text{\color{mathblue}descriptive features} \rightarrow \text{\color{mathblue}target features}\\
  \text{Goal: } & \text{find } f \text{ minimizing } error(\text{\color{mathblue}prediction}, \text{\color{mathblue}observed data})
\end{align*}
When we compare this approach to decision trees, we see:
\begin{itemize}
  \item Decision trees were initially developed for categorical features and then extended to continuous features.
  \item Regression followed the reverse path, which means it's most \textbf{suitable for continuous data}.
  \item Still, both are supervised learning techniques.
\end{itemize}

\subsection{Simple linear regression}


\subsection{Multiple descriptive features}

We now assume a feature consisting of multiple elements:
\begin{align*}\begin{aligned}
  \mathbb{M}_\mathbf{w}(\mathbf{d}) &= \mathbf{w}[0] + \mathbf{w}[1]\mathbf{d}[1] + \cdots + \mathbf{w}[m]\mathbf{d}[m]\\
  &= \mathbf{w}[0] + \sum_{j=1}^m \underbrace{\mathbf{w}[j]}_{\text{weight of the }i\text{-th feature }d[j]}\mathbf{d}[j]\\
  &= \sum_{j=0}^m \mathbf{w}[j]\underbrace{\mathbf{d}[j]}_{\text{with }d[0] = 1}\\
  &= \underbrace{\mathbf{w}\cdot\mathbf{d}}_{\text{dot product of vectors}}
\end{aligned}\end{align*}

In particular, this means we have $m+1$-dimensional vectors $\mathbf{d}_i$ and $\mathbf{w}$ with $m$ as the number of features. This notational convenience extends the normal feature vector $\mathbf{d}_i'$ by $\mathbf{d}_i[0] = 1$. With $n$ instances, we therefore have the error function:
\begin{align*}\begin{aligned}
  L_2(\mathbb{M}_\mathbf{w}, \mathcal{D}) &= \frac{1}{2}\sum_{i=1}^n (t_i - \underbrace{\mathbf{w}\cdot\mathbf{d}_i}_{= \mathbb{M}_\mathbf{w}(\mathbf{d}_i)})^2
\end{aligned}\end{align*}

With our now introduced multiple descriptive feature vector, we can write down the \textbf{sketch of} the overall \textbf{algorithm} of regression:
\begin{figure}[h]
  \begin{lstlisting}[
    style=pseudocode, morekeywords={Require, if, then, else, for, each, do, make, return, repeat, until},
    caption={Sketch of regression algorithm}, captionpos=b,
    label=lst:4_regr_sketch
  ]
  Require: set of descriptive features $\mathcal{D}$
  Require: learning rate $\alpha$ (controls how quickly algorithm converges)
  Require: function $\Delta_{error}$ (determines the direction in which to adjust given weight $\mathbf{w}[i]$ to move down the slope of an error surface determined by $\mathcal{D}$)
  Require: convergence criterion (indicating when the algorithm has been completed)
  
  $\mathbf{w}\leftarrow$ random starting point in weight space // randomly pick initial point
  repeat
    for each $\mathbf{w}[j]\in \mathbf{w}$ do
      // run downhill in steepest direction with speed $\alpha$
      $\mathbf{w}[j]\leftarrow \mathbf{w}[j] + \alpha\Delta_{error}(\mathcal{D}, \mathbf{w})[j]$
  until convergence occurs // stop when improvements become too small
  \end{lstlisting}
\end{figure}

For $\Delta_{error}$ we typically choose $\frac{\delta}{\delta \mathbf{w}[j]}L_2$. This means:
\begin{itemize}
  \item If the derivative is positive, lower the weight $\mathbf{w}[j]$,
  \item If the derivative is negative, increase the weight $\mathbf{w}[j]$,
  \item While $\alpha>0$ determines the speed in both cases.
  \item Line 10 in \ref{lst:4_regr_sketch} would then result in $\mathbf{w}[j]\leftarrow \mathbf{w}[j] - \alpha \sum_{i=1}^n(t_i-\mathbf{w}\cdot\mathbf{d}_i)\mathbf{d}_i[j]$
\end{itemize}
\subsection{Interpretation of results}
\subsection{Hanlding categorical features}
\subsection{Logistic regression}
\subsection{Extensions (non-linear and multinomial)}

